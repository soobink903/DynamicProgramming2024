\documentclass[11pt, aspectratio=169]{beamer}

\usepackage{amsmath, amsfonts, microtype, nicefrac, amssymb, amsthm, centernot}

\usepackage{pgfpages}

\usepackage{helvet}
\usepackage[default]{lato}
\usepackage{array}

\usefonttheme[onlymath]{serif}

\usepackage[utf8]{inputenc}
\usepackage[T1]{fontenc}
\usepackage{textcomp}
\usepackage{bm}

\usepackage{mathpazo}
\usepackage{hyperref}
\usepackage{multimedia}
\usepackage{graphicx}
\usepackage{multirow}
\usepackage{graphicx}
\usepackage{dcolumn}
\usepackage{bbm}
\newcolumntype{d}[0]{D{.}{.}{5}}

\usepackage{graphicx}
\usepackage[space]{grffile}
\usepackage{booktabs}

\usepackage{setspace}

\usepackage{transparent}


%%% FIGURES %%%
\usepackage{caption, subcaption}
\usepackage{booktabs, siunitx}
\usepackage{pgfplots} 
%\usepackage[outdir=./figures]{epstopdf}
\usepackage{float}
\usepackage{graphicx}
\usepackage[absolute, overlay]{textpos}
\usepackage{epstopdf}


%%% TIKZ %%%
\usepackage{tikz}
\usepackage{verbatim}
\usetikzlibrary{arrows.meta}
\usetikzlibrary{positioning}
\usetikzlibrary{bending}
\usetikzlibrary{snakes}
\usetikzlibrary{calc}
\usetikzlibrary{arrows}
\usetikzlibrary{decorations.markings}
\usetikzlibrary{shapes.misc}
\usetikzlibrary{matrix, shapes, arrows, fit, tikzmark}


%%% ALGORITHM %%%
\usepackage{algorithm}
\usepackage[noend]{algpseudocode}
\usepackage{multimedia}


%%% APPENDIX SLIDE NUMBERING %%%
\usepackage{appendixnumberbeamer}


%%% BEAMER BUTTON %%%
%\setbeamertemplate{button}{\tikz
	%\node[
	%	inner xsep = 2pt, 
	%	draw = structure!0, 
	%	fill = myblue, 
	%	rounded corners = 4pt]{\color{white} \tiny\insertbuttontext};
	%}


%%% COLORS %%%
\definecolor{blue}{RGB}{0,38,118}
\definecolor{red}{RGB}{213,94,0}
\definecolor{yellow}{RGB}{240,228,66}
\definecolor{green}{RGB}{0,158,115}

\definecolor{myred}{RGB}{163,32,45}
\definecolor{navyblue}{rgb}{0.05,0.2,0.70}
\definecolor{myblue}{RGB}{0,51,150}
\definecolor{myorange}{RGB}{255,140,0}
\definecolor{myref}{RGB}{160,160,160}
\definecolor{shock}{RGB}{0, 125, 34}%{50, 168, 82}

\definecolor{background}{RGB}{255,253,218}

% Define a new transparent color
\definecolor{trans}{rgb}{1,1,1}
\colorlet{trans}{black!20} % 0 percent opacity

\hypersetup{
  colorlinks=false,
  linkbordercolor = {white},
  linkcolor = {blue}
}

\setbeamercolor{frametitle}{fg=blue}
\setbeamercolor{title}{fg=black}
\setbeamertemplate{footline}[frame number]
\setbeamertemplate{navigation symbols}{} 
\setbeamertemplate{itemize items}{-}
\setbeamercolor{itemize item}{fg=blue}
\setbeamercolor{itemize subitem}{fg=blue}
\setbeamercolor{enumerate item}{fg=blue}
\setbeamercolor{enumerate subitem}{fg=blue}
\setbeamercolor{button}{bg=background, fg=blue,}

%\setbeamercolor{background canvas}{bg=background}


%%% FRAME TITLE %%%
\setbeamerfont{title}{series=\bfseries, parent=structure}
\setbeamerfont{frametitle}{series=\bfseries, parent=structure}


%%% TRANSITION FRAME %%%
\newenvironment{transitionframe}{
	\setbeamercolor{background canvas}{bg=blue}
	\begin{frame}
		\thispagestyle{empty}
		\addtocounter{framenumber}{-1}
		\vspace{42mm}
		\hspace{4mm} }{
		\begin{tikzpicture}
			\tikz \fill [white] (1,6) rectangle (20,10);
		\end{tikzpicture}
	\end{frame}
}


%%% OUTLINE %%%
\AtBeginSection[]
{
	\begin{frame}
       \frametitle{Roadmap of Talk}
       \tableofcontents[currentsection]
   \end{frame}
}
\setbeamercolor{section in toc}{fg=blue}
\setbeamercolor{subsection in toc}{fg=red}
\setbeamersize{text margin left=1em,text margin right=1em} 


%%% ENVIRONMENTS
\newenvironment{witemize}{\itemize\addtolength{\itemsep}{10pt}}{\enditemize}

\makeatother
\setbeamertemplate{itemize items}{\large\raisebox{0mm}{\textbullet}}
\setbeamertemplate{itemize subitem}{\footnotesize\raisebox{0.15ex}{--}}
\setbeamertemplate{itemize subsubitem}{\Tiny\raisebox{0.7ex}{$\blacktriangleright$}}

\setbeamertemplate{enumerate item}[default]
\setbeamertemplate{enumerate subitem}{\textbullet}
\makeatletter

% ITEMIZE SPACING:
% \usepackage{xpatch}
% \xpatchcmd{\itemize}
% {\def\makelabel}
% {\setlength{\itemsep}{0mm}\def\makelabel}
% {}
% {}


%%% PRETTY ENUMERATE %%%
% \usepackage{stackengine,xcolor}
% \newcommand\circnum[2]{\stackinset{c}{}{c}{.1ex}{\small\textcolor{white}{#2}}%
	% 	{\abovebaseline[-.7ex]{\Huge\textcolor{#1}{$\bullet$}}}}
% \newenvironment{myenum}
% {\let\svitem\item
	% 	\renewcommand\item[1][black]{%
		% 		\refstepcounter{enumi}\svitem[\circnum{##1}{\theenumi}]}%
	% 	\begin{enumerate}}{\end{enumerate}}
\usepackage{stackengine,xcolor,graphicx}
\newcommand\circnum[2]{\smash{\stackinset{c}{}{c}{.2ex}{\small\textcolor{white}{#2}}%
		{\abovebaseline[-1.1ex]{\Huge\textcolor{#1}{\scalebox{1.5}{$\bullet$}}}}}}
\newenvironment{myenum}
{\let\svitem\item
	\renewcommand\item[1][black]{%
		\refstepcounter{enumi}\svitem[\circnum{##1}{\theenumi}]}%
	\begin{enumerate}}{\end{enumerate}}

\newcommand{\notimplies}{\;\not\!\!\!\implies}



%%%%%%%%%%%%%%%%%%%%%%%%%%  TITLE   %%%%%%%%%%%%%%%%%%%%%%%%%%%%%%%%
\title[]{\\[8pt]
	{\large \color{blue} Dynamic Programming and Applications \\[5pt] \normalfont{Applications} \\[10pt] \normalfont{Lecture 7}}}
\author[Schaab]{Andreas Schaab}
\institute{}
\subject{}
\date{}



%%%%%%%%%%%%%%%%%%%%%%%%  BEGIN DOC   %%%%%%%%%%%%%%%%%%%%%%%%%%%%%%%
\begin{document}

%%% TIKZ %%% 
\tikzstyle{every picture}+=[remember picture]
%\everymath{\displaystyle}

\tikzset{   
	every picture/.style={remember picture,baseline},
	every node/.style={anchor=base,align=center,outer sep=1.5pt},
	every path/.style={thick},
}
\newcommand\marktopleft[1]{%
	\tikz[overlay,remember picture] 
	\node (marker-#1-a) at (-.3em,.3em) {};%
}
\newcommand\markbottomright[2]{%
	\tikz[overlay,remember picture] 
	\node (marker-#1-b) at (0em,0em) {};%
}
\tikzstyle{every picture}+=[remember picture] 
\tikzstyle{mybox} =[draw=black, very thick, rectangle, inner sep=10pt, inner ysep=20pt]
\tikzstyle{fancytitle} =[draw=black,fill=red, text=white]


\addtocounter{framenumber}{-1}
\thispagestyle{empty}
\maketitle 
\newpage




%%%%%%%%%%%%%%%%%%%%%%%%%%  SLIDE   %%%%%%%%%%%%%%%%%%%%%%%%%%%%%%%%
\begin{frame}{Outline}
\thispagestyle{empty}
\addtocounter{framenumber}{-1}

Part 1: Stochastic processes, Brownian motion, and stochastic differential equations
\begin{enumerate}
	\item Stochastic processes in continuous time
	\item Continuous time Markov chains
	\item Brownian motion
	\item Diffusion processes 
	\item Ito's Lemma
	\item Poisson processes
	\item The generator of a stochastic process
\end{enumerate}

\end{frame}


%%%%%%%%%%%%%%%%%%%%%%%%%%  SLIDE   %%%%%%%%%%%%%%%%%%%%%%%%%%%%%%%%
\begin{frame}{Outline}
\thispagestyle{empty}
\addtocounter{framenumber}{-1}

Part 2: Optimization with stochastic dynamics
\begin{enumerate}
	\item Stochastic neoclassical growth model
	\item Stochastic neoclassical growth with diffusion process
	\item Stochastic neoclassical growth with Poisson process
\end{enumerate}

\end{frame}


%%%%%%%%%%%%%%%%%%%%%%%%%%  SLIDE   %%%%%%%%%%%%%%%%%%%%%%%%%%%%%%%%
\begin{frame}{Real Business Cycles}
\begin{witemize}
\item Next semester, Yuriy will teach the Real Business Cycle model. This is basically the stochastic neoclassical growth model, estimated to match business cycle moments 

\item Recall that this model is efficient so we can look at the planning problem

\item Preferences are:
\begin{equation*}
	\mathbb E_0 \int_0^\infty e^{- \rho t} u(c_t) dt
\end{equation*}

\item And combining technologies and resource constraints yields:
\begin{equation*}
	dk_t = \Big[ f(k_t, z_t) - \delta k_t - c_t \Big] dt
\end{equation*}

\item Now suppose TFP follows a Feller process: $dz_t = \theta(\bar z - z_t) dt + \sigma \sqrt{z_t} dB_t$
\end{witemize}
\end{frame}


%%%%%%%%%%%%%%%%%%%%%%%%%%  SLIDE   %%%%%%%%%%%%%%%%%%%%%%%%%%%%%%%%
\begin{frame}{}
\begin{witemize}
\item With $dz_t = {\color{blue} \theta(\bar z - z_t) } dt + {\color{red} \sigma \sqrt{z_t} } dB_t$, the HJB is then given by:
\begin{equation*}
	\rho V(k, z) = \max_c \bigg\{ u(c) + \Big[ f(k, z) - \delta k - c \Big] V_k(k, z) + {\color{blue} \theta(\bar z - z) } V_z(k, z) + \frac{1}{2} {\color{red} \sigma^2 z } V_{zz}(k, z) \bigg\}
\end{equation*}

\item We have now seen 3 variants of this model with 3 different assumptions for the process $dz_t$

\item This model is the foundation for business cycle macro 
\end{witemize}
\end{frame}


%%%%%%%%%%%%%%%%%%%%%%%%%%  SLIDE   %%%%%%%%%%%%%%%%%%%%%%%%%%%%%%%%
\begin{frame}{AK Technology and Log Utility}
\begin{witemize}
\item Assume that $u(c_t) = \log c_t$ and $f(k_t, z_t) = z_t k_t$

\item Assuming that $dz_t$ follows a stationary diffusion process:
\begin{equation*}
	\rho V(k, z) = \max_c \bigg\{ \log c + \Big[ f(k, z) - \delta k - c \Big] V_k(k, z) + \mu(z) V_z(k, z) + \frac{1}{2} \sigma(z)^2 V_{zz}(k, z) \bigg\}
\end{equation*}

\item Show that the consumption policy function is:
\begin{equation*}
	c(k, z) = \rho k
\end{equation*}

\item As a result, model solution characterized by the two \textit{forward} equations:
\begin{align*}
	dk_t &= (z_t - \rho - \delta) k_t dt \\
	dz_t &= \mu(z_t) dt + \sigma(z_t) dB_t
\end{align*}

\end{witemize}
\end{frame}


%%%%%%%%%%%%%%%%%%%%%%%%%%  SLIDE   %%%%%%%%%%%%%%%%%%%%%%%%%%%%%%%%
\begin{frame}{}
\textbf{Proof:}
\begin{witemize}
\item Guess and verify:
\begin{equation*}
	V(k, z) = v(z) + A \log k
\end{equation*}

\item FOC: 
\begin{equation*}
	u'(c(k, z)) = V_k(k, z) 
	\quad \implies \quad
	\frac{1}{c(k, z)} = A \frac{1}{k}
	\quad \implies \quad
	c(k, z) = \frac{1}{A} k
\end{equation*}

\item Plug back into HJB: 
\begin{equation*}
\rho v(z) + \rho A \log k = \log \frac{1}{A} + \log k + (z - \frac{1}{A} - \delta) k A \frac{1}{k} + \mu(z) v'(z) + \frac{1}{2} \sigma(z)^2 v''(z)
\end{equation*}

\item Collect terms in $\log k$ and confirm $\rho A = 1$. Solve ODE for $v(z)$!

\item Economics: log preferences $\implies$ income and substitution effects of future $z_t$ changes cancel $\implies$ constant savings rate $\rho$
\end{witemize}
\end{frame}







\end{document}
